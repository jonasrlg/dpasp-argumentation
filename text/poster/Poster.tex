\documentclass[final]{beamer}

% ====================
% Packages
% ====================

\usepackage[T1]{fontenc}
\usepackage{lmodern}
\usepackage[orientation=portrait,size=a0,scale=1.25]{beamerposter}
\usetheme{gemini}
\usecolortheme{nott}
\usepackage{graphicx}
\usepackage{booktabs}
\usepackage{tikz}
\usepackage{pgfplots}
\pgfplotsset{compat=1.14}
\usepackage{anyfontsize}
\usepackage{natbib}
\usepackage{amsmath}suport
\usepackage{amsthm}
\usepackage{amssymb}
\usetikzlibrary{trees, positioning, arrows.meta, calc, shapes.arrows}
% ====================
% TikZ Styles
% ====================
\tikzstyle{sum}=[draw,circle,inner sep=4pt, cyan,thick, minimum size=20pt]
\tikzstyle{and}=[draw,circle,inner sep=4pt, orange,thick, minimum size=20pt]
\tikzstyle{lit}=[inner sep=6pt,black, minimum size=15pt, font=\sffamily]
\tikzstyle{arg}=[draw,circle,thick,minimum size=32pt,font=\large\sffamily,fill=white]
\tikzstyle{support}=[->,>=Latex,line width=1.5pt,color=blue!70]
\tikzstyle{defeat}=[->,>=Latex,line width=1.5pt,color=red!70]

% ====================
% Lengths
% ====================

\newlength{\sepwidth}
\newlength{\colwidth}
\setlength{\sepwidth}{0.025\paperwidth}
\setlength{\colwidth}{0.45\paperwidth}

\newcommand{\separatorcolumn}{\begin{column}{\sepwidth}\end{column}}

% ====================
% Title
% ====================

\title{Knowledge Compilation for Probabilistic Answer Set Programming}

\author{Jonas R. L. Gonçalves \and Denis D. Mauá}

\institute{Computer Science Department, Institute of Mathematics and Statistics\\University of São Paulo, São Paulo, Brazil}

% ====================
% Footer
% ====================

\footercontent{
  \href{mailto:jonasrlg@ime.usp.br}{jonasrlg@ime.usp.br} \hfill
  Tractable Probabilistic Modeling --- Poster Session \hfill
  \href{mailto:ddm@ime.usp.br}{ddm@ime.usp.br}}

% Create bibliography file inline

% ====================
% Body
% ====================

\begin{document}

\begin{frame}[t]
\begin{columns}[t]
\separatorcolumn

\begin{column}{\colwidth}

  \begin{block}{Probabilistic Answer Set Programming (PASP)}

    \textbf{PASP} extends answer set programming with probabilistic facts, enabling declarative modeling of hybrid logical and stochastic knowledge. This makes it possible to perform inference over complex combinatorial domains by reasoning simultaneously about structure and uncertainty.

    \heading{Why PASP?}
    \begin{itemize}
      \item Models relational processes such as hidden Markov models or planning scenarios with latent variables.
      \item Separates knowledge representation from inference, supporting re-use of programs with different evidence.
      \item Supports multiple semantics for uncertainty: \textbf{max-entropy} distributions and \textbf{credal} (upper/lower) bounds.
    \end{itemize}

    \heading{Illustrative Fragment}
    \begin{center}
  \begin{minipage}{\linewidth}
    \ttfamily
    0.5::rain.\quad 0.4::sprinkler.\\
    wet :- rain.\\
    wet :- sprinkler.\\
    call :- wet.
  \end{minipage}
\end{center}




    \heading{Inference Tasks}
    \begin{itemize}
      \item Compute probability of atoms (e.g., $\mathbb{P}(\texttt{wet})$) under max-entropy semantics.
      \item Derive credal bounds to capture epistemic uncertainty about competing stable models.
      \item Answer probabilistic queries conditioned on evidence without re-grounding the program.
    \end{itemize}

  \end{block}

  \begin{block}{Knowledge Compilation Goals}

    Compilation replaces repeated search by an \textbf{offline} transformation of the grounded PASP program into a circuit that supports tractable \textbf{weighted model counting (WMC)}.

    \heading{Workflow}
    \begin{enumerate}
      \item Ground and simplify the PASP rules while preserving probabilistic annotations.
      \item Build a structured circuit whose leaves carry literal weights and whose internal gates encode logical constraints.
      \item Evaluate the circuit once per query or evidence set, amortizing inference over many questions.
    \end{enumerate}

    \heading{Benefits}
    \begin{itemize}
      \item Complexity becomes proportional to circuit size rather than the exponential number of stable models.
      \item Circuit evaluation can be parallelized and differentiated, enabling learning and sensitivity analysis.
      \item Shared compilation pipeline for both max-entropy probabilities and credal bounds.
    \end{itemize}

    \begin{figure}
      \centering
      \includegraphics[width=\linewidth]{diagram.pdf}
      \caption{Knowledge compilation pipeline for PASP, adapted from Manhaeve \emph{et al.} (2018) \cite{manhaeve2018deepproblog}.}
    \end{figure}

  \end{block}

\end{column}

\separatorcolumn

\begin{column}{\colwidth}

  \begin{block}{Structured Circuits for Probabilistic Inference}

    \textbf{Weighted model counting} turns probabilistic reasoning into a sum over satisfying:
    \[
      \mathrm{WMC}(F) = \sum_{I \in \mathcal{M}(F)} \prod_{\ell \in I} w(\ell).
    \]

    \heading{Circuit Properties}
    \begin{itemize}
      \item \textbf{Decomposability}: inputs of each product node depend on disjoint variable sets.
      \item \textbf{Determinism}: inputs of sum nodes correspond to mutually exclusive cases.
      \item \textbf{Smoothness}: inputs of sum nodes mention the same variables, enabling weight sharing.
    \end{itemize}

    \heading{Why they matter}
    These structural guarantees allow linear-time evaluation of evidence and queries, certified reuse across multiple PASP tasks, and compatibility with model counting.

    \begin{figure}
      \centering
      \begin{tikzpicture}[scale=1.3]
        \node[sum] (s0) at (0,0) {$+$};
        \node[and] (p1) at (-2,-1.6) {$\times$};
        \node[and] (p2) at (2,-1.6) {$\times$};
        \node[sum] (s1) at (-3.2,-3.1) {$+$};
        \node[lit] (s2) at (-0.8,-3.1) {$b$};
        \node[lit] (la) at (-4.2,-4.6) {$a$};
        \node[lit] (lna) at (-2.2,-4.6) {$\neg a$};
        \node[lit] (ld) at (2.8,-3.1) {$a$};
        \node[lit] (le) at (4.2,-3.1) {$\neg b$};

        \draw (s0) -- (p1);
        \draw (s0) -- (p2);
        \draw (p1) -- (s1);
        \draw (p1) -- (s2);
        \draw (s1) -- (la);
        \draw (s1) -- (lna);
        \draw (p2) -- (ld);
        \draw (p2) -- (le);
      \end{tikzpicture}
      \caption{Smooth, decomposable, deterministic circuit fragment enabling tractable WMC.}
    \end{figure}

  \end{block}

  \begin{block}{Bipolar Argumentation Frameworks}

    \textbf{Argumentation} supports explanations and conflict management in PASP by reasoning over argumentative structures derived from stable models.

    \heading{Definition \cite{egly2008baf}}
    A \textbf{BAF} is a triple $(A, R_d, R_s)$ with arguments $A$, defeat edges $R_d$, and support edges $R_s$. Defeat is propagated through chains that alternate supports and a single defeat, enabling rich interactions beyond simple attacks.

    \heading{Example}
    \begin{figure}
      \centering
      \begin{tikzpicture}[node distance=3.0cm]
        % tornar todos os nós do mesmo tamanho
        \tikzstyle{arg}=[draw,circle,thick,minimum size=38pt,font=\large\sffamily,fill=white]
        
        % espaçar mais os nós
        \node[arg] (claim) {Claim};
        \node[arg] (study) [below left=2.4cm and 2.2cm of claim] {Data};
        \node[arg] (expert) [below right=2.4cm and 2.2cm of claim] {Expert};
        \node[arg] (counter) [below=6.2cm of claim] {Counter};


        \draw[support] (study) -- node[left, font=\footnotesize\sffamily, yshift=4pt] {supports} (claim);
        \draw[support] (expert) -- node[right, font=\footnotesize\sffamily, yshift=4pt] {supports} (claim);
        \draw[defeat] (counter) -- node[right, font=\footnotesize\sffamily] {defeats} (claim);
        \draw[defeat] (study) -- node[left, font=\footnotesize\sffamily] {defeats} (counter);
      \end{tikzpicture}
      \caption{Support edges (blue) reinforce the claim, while defeat edges (red) capture rebuttals.}
    \end{figure}

    \heading{Integration with PASP}
    \begin{itemize}
      \item Encode arguments as rules; supports arise from derivations, defeats from conflicting explanations.
      \item Preferred extensions summarise coherent probabilistic explanations for observed evidence.
      \item Argumentation-aware compilation guides model exploration and explanation ranking.
    \end{itemize}

  \end{block}

  % \begin{block}{Compilation Pipeline}

  %   \begin{figure}
  %     \centering
  %     \includegraphics[width=0.7\linewidth]{diagram.pdf}
  %     \caption{Knowledge compilation pipeline for PASP, adapted from Manhaeve \emph{et al.} (2018) \cite{manhaeve2018deepproblog}.}
  %   \end{figure}

  %   \heading{From Programs to Circuits}
  %   \begin{itemize}
  %     \item Grounding produces a propositional representation while preserving probabilistic annotations.
  %     \item Deterministic, smooth, decomposable circuits enable exact WMC and algebraic generalisations.
  %     \item Argumentation overlays remain consistent with compiled knowledge, providing transparent explanations.
  %   \end{itemize}

  % \end{block}

  \begin{block}{References}
    \bibliographystyle{plain}
    \bibliography{poster}
  \end{block}

\end{column}

\separatorcolumn

\end{columns}
\end{frame}

\end{document}

