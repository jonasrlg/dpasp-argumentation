\documentclass{article}
\usepackage[utf8]{inputenc}
\usepackage[T1]{fontenc}
\usepackage{amsmath}
\usepackage{amssymb}
\usepackage{graphicx}
\usepackage{hyperref}
\usepackage{natbib}
\usepackage{booktabs}
\usepackage{xcolor}

\title{Enhancing Argumentation Structure Parsing with Deep Probabilistic Answer
Set Programming}
\author{Jonas Rodrigues Lima Gonçalves \\ \textit{Advisee of Prof. Denis
Deratani Mauá} \and Denis Deratani Mauá \\ \textit{Advisor}}

\date{\today}

\begin{document}

\maketitle

\section{Introduction}

Argumentation Mining (AM) is a significant area within Natural Language
Processing (NLP) focused on extracting argumentative structures from text. Stab
and Gurevych have contributed significantly to this field by proposing a
pipeline approach for parsing argumentation structures in persuasive essays.
Their method involves segmenting the extraction process into several sub-steps,
including identifying argument components, classifying their types, and
identifying argumentative relations. A crucial final step in their pipeline
utilizes Integer Linear Programming (ILP) to enforce coherence between the
outputs of these sub-steps and generate a globally consistent argumentation
graph.

While the ILP approach offers a way to integrate local predictions into a global
structure, it can face limitations in terms of capturing complex probabilistic
dependencies and being seamlessly integrated with end-to-end differentiable
learning. Recent advancements in neural-symbolic reasoning, particularly in
Deep Probabilistic Logic Programming frameworks like Problog, Scallop and dPASP,
offer promising alternatives. dPASP extends Probabilistic Answer Set Programming
(PASP) with neural predicates, as probabilistic choices, and tight integration
with deep learning frameworks like PyTorch, enabling end-to-end training through
automatic differentiation.

This thesis project aims to \textbf{investigate the potential of replacing the
integer programming-based coherence enforcement step in Stab and Gurevych's
argumentation parsing pipeline with Scallop and dPASP}. By leveraging both
Scallop's and dPASP's capabilities for probabilistic modeling and integration
with neural networks, this project seeks to develop a more flexible and
potentially more accurate approach to ensuring the coherence of extracted
argumentation structures.

The overarching goal is to contribute to the development of more effective
neural-symbolic approaches for argumentation mining and to evaluate the
capabilities of next-generation StarAI (Statistical Relational Learning)
systems.

\section{Objectives and Goals}

The primary objective of this thesis project is to \textbf{design, implement,
and evaluate a logic-based module for enforcing coherence in an argumentation
structure parsing pipeline, drawing inspiration from the work of Stab and
Gurevych}. This will be done in two phases, initially focusing on Scallop and
subsequently exploring dPASP.

The specific goals of this project include:

\begin{itemize}
    \item \textbf{Understanding the Existing Pipeline:} Thoroughly analyze the
    argumentation structure parsing pipeline proposed by Stab and Gurevych, with
    a specific focus on the role and implementation of the integer linear
    programming step. This includes understanding the input to this step
    (outputs of the component identification, classification, and relation
    identification models) and its output (the coherent argumentation graph).

    \item \textbf{Implementing a Scallop-based Coherence Model (Phase 1):}
    Develop a logic program in Scallop that can take the outputs
    of the preceding pipeline stages (or their representations) as input and
    enforce coherence rules for the argument structure. This will involve
    collaborating with Vinícius and leveraging his work on Scallop plugins.

    \item \textbf{Training Necessary Models (Phase 1):} Train the machine
    learning models required for argument component identification, node
    classification, and edge classification using the Stab and Gurevych
    dataset. This step is crucial for providing input to the Scallop-based
    coherence module.

    \item \textbf{Integrating the Scallop Module (Phase 1):} Implement the
    integration of the Scallop module into the existing argumentation parsing
    pipeline, potentially using publicly available resources. This will involve
    handling the interface between the machine learning classifiers and the
    Scallop framework.

    \item \textbf{Evaluating the Scallop-enhanced Pipeline (Phase 1):} Evaluate
    the performance of the modified argumentation parsing pipeline, comparing it
    to the original approach that uses integer programming. This evaluation
    will likely involve using the argumentation structure annotated corpus and
    standard evaluation metrics for argumentation mining.

    \item \textbf{Conceptualizing a dPASP-based Coherence Model (Phase 2 -
    Tentative):} Explore the development of a probabilistic logic program in
    dPASP that can take the probabilistic outputs of the preceding pipeline
    stages (or their representations) as input and reason about the coherent
    structure of the argument. This may involve defining neural predicates.

    \item \textbf{Implementing and Evaluating the dPASP Module (Phase 2 -
    Tentative):} Implement the dPASP module and compare its performance with
    the Scallop-based approach, potentially leveraging advancements in dPASP
    2.0, such as compilation techniques using SDDs and decision sdDNNFs.

    \item \textbf{Analyzing the Results:} Analyze the strengths and weaknesses
    of both the Scallop and (if implemented) dPASP-based approaches,
    identifying potential improvements and discussing the implications for the
    field of argumentation mining and neural-symbolic reasoning.
\end{itemize}

\section{Planning Major Activities}

The project will be conducted through the following major activities:

\begin{itemize}
    \item \textbf{Phase 1: Scallop Implementation and Evaluation (Months 1-5):}
    \begin{itemize}
        \item \textbf{Month 1:} In-depth study of Stab and Gurevych's work and
        familiarization with the argumentation structure annotated corpus.
        \item \textbf{Month 2:} Comprehensive review of the Scallop framework
        and collaboration with Vinícius on the plugin development.
        \item \textbf{Months 3-4:} Training the BERT-based models for argument
        component identification, node classification, and edge classification.
        \item \textbf{Month 5:} Implementation and integration of the
        Scallop-based coherence module and evaluation of the enhanced pipeline.
    \end{itemize}

    \item \textbf{Phase 2: dPASP Exploration and Comparison (Months 6-7 -
    Tentative):}
    \begin{itemize}
        \item \textbf{Month 6:} Study of dPASP 2.0 advancements (compilation
        with SDDs and decision sdDNNFs) and conceptualization of a
        dPASP-based coherence model.
        \item \textbf{Month 7:} Implementation of the dPASP module and
        comparative evaluation with the Scallop-based approach (if dPASP 2.0
        is sufficiently mature).
    \end{itemize}

    \item \textbf{Final Phase: Analysis and Thesis Writing (Months 7-8):}
    \begin{itemize}
        \item In-depth analysis of the results from both phases.
        \item Discussion of the findings and their implications.
        \item Writing and finalizing the thesis document.
    \end{itemize}
\end{itemize}

\section{Methodology}

This project will primarily employ a stratified-based approach for the initial
implementation using Scallop, with a potential secondary exploration of a
non-stratified approach using dPASP. The core methodology will involve:

\begin{itemize}
    \item \textbf{Replication and Adaptation:} Starting with the well-defined
    pipeline of Stab and Gurevych, the project will focus on replicating the
    overall structure while specifically replacing the ILP-based coherence
    enforcement with logic-based reasoning in Scallop.

    \item \textbf{Logic Programming with Scallop:} The coherence of the
    extracted argument components and relations will be modeled using Scallop,
    a highly efficient deductive reasoning system. Probabilistic logic rules
    will be defined to capture the constraints and desired properties of a
    coherent argumentation structure.

    \item \textbf{Integration with Machine Learning Models:} The Scallop module
    will be integrated with the output of pre-trained (or trained as part of
    this project) machine learning models for argument component
    identification, classification, and relation identification.

    \item \textbf{Neural-Symbolic Exploration with dPASP (Tentative):} If time
    and the maturity of dPASP 2.0 allow, a parallel or subsequent
    implementation using dPASP will be explored. This would involve defining
    probabilistic logic programs, potentially with neural predicates, to model
    coherence.

    \item \textbf{Experimental Evaluation:} The performance of the
    Scallop-enhanced pipeline will be rigorously evaluated on the annotated
    corpus of persuasive essays by Stab and Gurevych, using standard
    evaluation metrics. If a dPASP-based approach is implemented, its
    performance will also be evaluated and compared.
\end{itemize}

\section{Expected Outcomes and Contributions}

This thesis project is expected to yield the following outcomes and
contributions:

\begin{itemize}
    \item \textbf{A Scallop-based module for enforcing coherence in
    argumentation structure parsing.} This module will demonstrate the
    application of efficient logic programming to a challenging NLP task.

    \item \textbf{An evaluation of the effectiveness of Scallop compared to
    integer programming for ensuring coherence in argumentation parsing.} This
    comparison will provide insights into the strengths and weaknesses of
    stratified-based approaches for this task.

    \item \textbf{Potential improvements in the accuracy and efficiency of
    argumentation structure parsing.} By leveraging the efficiency of Scallop,
    the project may lead to a more scalable approach to coherence enforcement.

    \item \textbf{A deeper understanding of the integration of machine learning
    and logical reasoning for argumentation mining.} The project will
    contribute to the field by exploring a practical application of probabilistic
    logic programming for coherence enforcement.

    \item \textbf{A documented implementation and evaluation of the proposed
    approach}, which could serve as a foundation for future research in this
    area, potentially including a comparison with dPASP.
\end{itemize}

\section{Timeline}

\begin{tabular}{|l|l|}
\hline
\textbf{Time Period} & \textbf{Major Activities} \\
\hline
Month 1 & Literature Review and Background Study \\
\hline
Month 2 & Scallop Framework Study and Collaboration \\
\hline
Months 3-4 & Training Machine Learning Models \\
\hline
Month 5 & Scallop Module Implementation and Evaluation \\
\hline
Month 6 & dPASP Study and Conceptualization (Tentative) \\
\hline
Month 7 & dPASP Implementation and Comparison (Tentative) \\
\hline
Months 7-8 & Analysis and Thesis Writing \\
\hline
\end{tabular}

\section{Required Resources}

\begin{itemize}
    \item \textbf{Computational Resources:} Access to adequate computational
    resources, including GPUs, for training and evaluating deep learning models.

    \item \textbf{Software Resources:} Python programming environment, PyTorch
    for deep learning, and the Scallop framework. Access to additional
    libraries for NLP tasks. If Phase 2 is pursued, access to the dPASP
    framework.

    \item \textbf{Data Resources:} The annotated corpus of persuasive essays
    created by Stab and Gurevych.

    \item \textbf{Academic Resources:} Access to relevant scientific literature.
\end{itemize}

\section{Risks and Mitigation Strategies}

\begin{itemize}
    \item \textbf{Risk:} Technical challenges in integrating the Scallop module
    with the machine learning pipeline.
    \textbf{Mitigation:} Start with a clear interface definition and modular
    design. Thoroughly test the integration at each step.

    \item \textbf{Risk:} The Scallop-based approach may not perform as well as
    the original ILP approach.
    \textbf{Mitigation:} Focus the evaluation on understanding the specific
    strengths and weaknesses of the logic-based approach.

    \item \textbf{Risk:} Time constraints may prevent the full exploration of a
    dPASP-based comparison.
    \textbf{Mitigation:} Prioritize the Scallop implementation and evaluation.
    The dPASP phase will be contingent on the progress and available time.

    \item \textbf{Risk:} Difficulty in training accurate machine learning
    models for the initial stages of the pipeline.
    \textbf{Mitigation:} Leverage pre-trained BERT models and focus on fine-tuning
    them on the Stab and Gurevych dataset.
\end{itemize}

\section{Conclusion}

This thesis project will investigate the application of logic programming,
specifically using the Scallop framework, for enhancing argumentation structure
parsing. By replacing the integer linear programming step in Stab and Gurevych's
pipeline with a Scallop-based approach, the project aims to leverage the
efficiency and declarative power of probabilistic logic programming for ensuring
coherence. A secondary exploration of dPASP may be conducted depending on the
progress of related work. The expected outcomes include a novel implementation,
empirical evaluation, and insights into the integration of machine learning and
logical reasoning for this challenging NLP task. The results of this project
could potentially lead to more efficient and effective approaches to
argumentation mining.

\bibliographystyle{apalike}
\bibliography{references}

\end{document}
