\chapter{Conclusion}

We've presented novel methods for Non-Incremental Probabilistic Answer
Set Programming Knowledge Compilation,
alongside theoretical results demonstrating their potential for efficient
compilation. A key innovation in our approach lies in its Non-Incremental
nature: we decompose the original program into disjoint subsets, compile
each independently, and then conjoin their respective circuits to form
the program's final representation. Furthermore, we've adapted bottom-up
KC to effectively handle PASP-specific constraints, including
cardinality constraints and probabilistic semantics.

Overall, our methods demonstrate potential for significant improvements in
the efficiency and scalability of PASP inference, because it avoids
introducing auxiliary variables during compilation. By moving beyond
standard top-down pipelines and exploring alternative circuit compilation,
we enable the generation of considerably more intricate circuits. This
enhanced compilation capability, in turn, allows Neuro-Symbolic systems
to encode more complex real-world constraints. This deep integration of
neural perception with robust PASP reasoning can be further leveraged
by recent advances in the encoding of circuits on GPUs \citep{maene2024klay}.

Possible future works include experimental validation of the unconstrained
compilation approach proposed in this work, using recent techniques for
re-structuring a circuit, \citep{zhang2025}, to trade-off flexibility
during compilation with constraint enforcement after compilation for more
efficient inference. Another possibly interesting direction is to explore
the relation of the Non-Incremental methods with compositional ones,
\citep{dal2021compositional}, where the latter approaches the Weighted Model
Counting (WMC) tasks by dividing the problem into smaller subproblems,
similar in spirit to the Non-Incremental approach.
