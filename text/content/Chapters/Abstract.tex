%!TeX root=../tese.tex
%("dica" para o editor de texto: este arquivo é parte de um documento maior)
% para saber mais: https://tex.stackexchange.com/q/78101

% As palavras-chave são obrigatórias, em português e em inglês, e devem ser
% definidas antes do resumo/abstract. Acrescente quantas forem necessárias.
\palavraschave{Programação Lógico Probabilística, Compilação de Conhecimento,
Circuito Probabilísticos, Mineração de Argumentos}

\keywords{Probabilistic Logic Programming, Knowledge Compilation,
Probabilistic Circuits, Argument Mining}

% O resumo é obrigatório, em português e inglês. Estes comandos também
% geram automaticamente a referência para o próprio documento, conforme
% as normas sugeridas da USP.
\abstract{
Argumentation mining is a complex task that has been approached more
recently using purely connectionist methods. These methods aim to extract
arguments from text data and represent them in a structured format. However,
these methods often lack the ability to handle uncertainty and probabilistic
reasoning, and usually require large amounts of labeled data for training,
which often are not available in many real-world scenarios.

In order to address these limitations, frameworks for modeling and reasoning
about arguments have been developed. These frameworks model the problem via
a Neuro-Symbolic approach, either using Integer Linear Programming or Probabilistic
Logic Programming (PLP). While Integer Linear Programming is a well-known mathematical
optimization technique that can be used to model and solve complex
decision-making problems, the integration of such methods to constrain learning
and inference is a challenging task. On the other hand, Probabilistic Logic
Programming stands itself as a powerful tool with declarative semantics and
more ``readable'' syntax for non-experts.

As of the time of writing, current PLP frameworks for modeling Argumentation
Mining focused on using stratified programs, which largely restrict the
expressiveness of the different argumentation problems one may desire to
represent. Thus, we propose to model this problem using Probabilistic Answer
Set Programming (PASP), a framework that combines the expressiveness of ASP
with the probabilistic reasoning capabilities of PLP. Furthermore, in order
to be able to use this PASP framework for scalable Neuro-Symbolic learning,
we explore different state-of-the-art Knowledge Compilation (KC) techniques
of the language, which are able to encode PASP programs into circuits that
can be encoded as computational graphs in a variety of autodifferentiable
frameworks, such as PyTorch or Jax.
}

\resumo{
A mineração de argumentação é uma tarefa complexa que tem sido abordada mais
recentemente utilizando métodos puramente conexionistas. Esses métodos têm como
objetivo extrair argumentos de dados textuais e representá-los em um formato
estruturado. No entanto, esses métodos frequentemente carecem da capacidade de
lidar com incertezas e raciocínio probabilístico, além de geralmente exigirem
grandes quantidades de dados rotulados para treinamento, os quais muitas vezes
não estão disponíveis em muitos cenários do mundo real.

Para lidar com essas limitações, foram desenvolvidos frameworks para modelar e
realizar raciocínio sobre argumentos. Esses frameworks modelam o problema por
meio de uma abordagem Neuro-Simbólica, utilizando Programação Linear Inteira ou
Programação Lógica Probabilística (PLP). Enquanto a Programação Linear Inteira é uma
técnica de otimização matemática bem conhecida que pode ser usada para modelar
e resolver problemas complexos de tomada de decisão, a integração de tais
métodos para restringir aprendizado e inferência é uma tarefa desafiadora. Por
outro lado, a Programação Lógica Probabilística se destaca como uma ferramenta
poderosa com semântica declarativa e uma sintaxe mais ``legível'' para não
especialistas.

No momento da redação deste trabalho, os frameworks atuais de PLP para modelar
a Mineração de Argumentação focaram no uso de programas estratificados, o que
restringe amplamente a expressividade dos diferentes problemas de argumentação
que se deseja representar. Assim, propomos modelar esse problema utilizando a
Programação Lógica com Conjuntos de Respostas Probabilística (PASP), um
framework que combina a expressividade da ASP com as capacidades de raciocínio
probabilístico da PLP. Além disso, para ser capaz de usar esse framework PASP
para aprendizado Neuro-Simbólico escalável, exploramos diferentes técnicas
modernas de Compilação de Conhecimento (KC) da linguagem, que são capazes de
codificar programas PASP em circuitos que podem ser representados como grafos
computacionais em uma variedade de frameworks autodiferenciáveis, como PyTorch
ou Jax.
}
