%!TeX root=../tese.tex
%("dica" para o editor de texto: este arquivo é parte de um documento maior)
% para saber mais: https://tex.stackexchange.com/q/78101

% As palavras-chave são obrigatórias, em português e em inglês, e devem ser
% definidas antes do resumo/abstract. Acrescente quantas forem necessárias.
\palavraschave{Programação Lógico Probabilística, Compilação de Conhecimento,
Circuito Probabilísticos}

\keywords{Probabilistic Logic Programming, Knowledge Compilation,
Probabilistic Circuits}

% O resumo é obrigatório, em português e inglês. Estes comandos também
% geram automaticamente a referência para o próprio documento, conforme
% as normas sugeridas da USP.
\abstract{
Advances in Probabilistic Logic Programming (PLP) and Probabilistic Inference (PI) have
demonstrated that Knowledge Compilation (KC) techniques, which compile probabilistic
logic programs into Answer Set Programs (ASP), are essential for
fast and exact inference. Despite the close relationship between Circuit Theory
and the complexity of logical inference, it is possible to exploit the structure
of PLPs to construct more succinct circuits that can efficiently answer
probabilistic queries.

Although previous research has shown that the compilation of stratified programs
can be achieved using various techniques, such as $T_P$-consequence or Loop
Formulas, little has been explored in the context of PASP compilation,
apart from top-down compilation approaches that translate the program into a
Conjunction Normal Form (CNF) and subsequently apply knowledge compilation techniques.
Decomposable Negation Normal Form (DNNF) compilation-based methods have also been
investigated.

Drawing inspiration from successful works on the compilation of stratified
PLP languages, this research aims to leverage the tractability and bottom-up
nature of a special class of Arithmetic Circuits (AC), called Probabilistic
Sentential Decision Diagrams (PSDDs), to compile PASP programs into PSDDs under the
\textit{Max Entropy} and \textit{Credal} semantics. Furthermore, this bottom-up
approach enables structural optimizations through modifications to the variable
tree that governs the structure of the circuits, resulting in more succinct
representations.

Finally, PSDDs are capable of performing exact inference in polynomial time and
support numerous algorithms for learning and regularization, specifically designed
for use in Probabilistic Circuits (PCs). This further underscores the potential of
PSDDs as a tractable representation for PLPs and highlights the need to explore
efficient heuristics for dynamically modifying the variable tree structure when
compilingPASP programs into PSDDs.
}

\resumo{
Avanços em Programação Lógica Probabilística e Inferência Probabilística
demonstraram que técnicas de Compilação de Conhecimento, as quais compilam
programas lógicos probabilísticos em Programas de Conjunto de Respostas,
são essenciais para inferência rápida e exata. Apesar da estreita relação
entre a Teoria de Circuitos e a complexidade da inferência lógica, é possível
explorar a estrutura dos Programas Lógicos Probabilísticos para construir
circuitos mais concisos que podem responder eficientemente a consultas
probabilísticas.

Embora pesquisas anteriores tenham mostrado que a compilação de programas
estratificados pode ser alcançada usando várias técnicas, como a
consequência do operador $T_P$ ou \textit{Loop Formulas}, pouco foi
explorado no contexto da compilação de Programas de Conjunto de Respostas
Probabilísticos, à parte de abordagens de compilação top-down (de cima
para baixo) que traduzem o programa em uma Forma Normal Conjuntiva e
subsequentemente aplicam técnicas de compilação de conhecimento. Métodos
baseados na compilação para Forma Normal Negativa Decomponível também
foram investigados.

Inspirada em trabalhos bem-sucedidos sobre a compilação de linguagens de
Programação Lógica Probabilística estratificada, esta pesquisa visa
alavancar a tratabilidade e a natureza bottom-up (de baixo para cima) de
uma classe especial de Circuitos Aritméticos, chamados Diagramas de
Decisão Sentencial Probabilísticos, para compilar Programas de Conjunto
de Respostas Probabilísticos nestes diagramas sob a semântica de
Entropia Máxima e Credal. Além disso, essa abordagem bottom-up permite
otimizações estruturais por meio de modificações na árvore de variáveis
que governa a estrutura dos circuitos, resultando em representações mais
concisas.

Finalmente, os Diagramas de Decisão Sentencial Probabilísticos são capazes
de realizar inferência exata em tempo polinomial e suportam inúmeros
algoritmos para aprendizado e regularização, especificamente projetados
para uso em Circuitos Probabilísticos. Isso reforça ainda mais o potencial
dos Diagramas de Decisão Sentencial Probabilísticos como uma representação
tratável para Programação Lógica Probabilística e destaca a necessidade de
explorar heurísticas eficientes para modificar dinamicamente a estrutura
da árvore de variáveis ao compilar Programas de Conjunto de Respostas
Probabilísticos nos Diagramas de Decisão Sentencial Probabilísticos.
}
