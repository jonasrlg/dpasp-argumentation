% Appendix Circuits and Knowledge Compilation

\chapter{Appendix Circuits and Knowledge Compilation}

\label{ap:circuits}

%---------------------------------------------------------------

Through years of theoretical and experimental studies, the last 
decades have proven that \acl{KC} is a key focus of research for
dealing with computational intractability of general 
propositional reasoning \citep{Darwiche_2002}. This tractability
is achieved via two steps:

\begin{enumerate}
    \item First, the \textit{off-line} compilation of a 
        (logical) propositional theory into a specified
        language, such as \ac{NNF} (that will be discussed in 
        more detail in the next sections);
    \item Then the \textit{on-line} answering of large number
        of logical queries, in polynomial time w.r.t. the size
        of compiled theory.
\end{enumerate}

The motivation of this procedure is to push as much of the 
computational overhead into the \textit{off-line} phase; which
is amortized over the \textit{on-line} phase, due to the 
polynomial complexity of the queries \citep{Darwiche_2002}. 
Therefore, the challenge of this type of this area itself is not
so different of others fields in Artificial Intelligence: the 
Min-Max nature of trying to solve or approximate problems that
are known to be hard (\ac{NP}). 

In the specific case of \acl{KC}, the Min-Max relation is due to
the fact that we want to minimize the overhead of the compiling
a theory, which is a step know to be hard, otherwise it would be
possible to solve \ac{SAT} in polynomial time (an hypothesis 
that we find to be untrue, on the premises of this thesis). As
we will see, minimizing the overhead of this first step will not
necessarily produce a circuit that allows fast answers to the 
respective set o queries. Another possible scenario is that the 
queries are indeed able to be answered in polynomial time w.r.t. 
the size of the compilation's product, but this product has an 
exponential size, due to not exploring intrinsic logical 
relations of the theory.

One can not claim, at least in the state of the research that
is found today, that there is a language that minimizes the 
compilation overhead and enables fast answering of queries for 
any arbitrary query. Among other reasons, we have that the 
transitional phase nature of logical problems, such as \ac{SAT}
\citep{gent1994sat}, renders this \textit{No free lunch} 
\citep{Wolpert_No_Free_Lunch} characteristic to arbitrary 
optimization of general theories. On the other hand, better 
algorithms and language classes are extensively developed, 
especially when we reduce the scope of the target theories. 
Reducing the class of studied theories also makes it possible 
to narrow the set of languages that are fit for the compilation,
and enables us to determine which of them consistently perform
better within the specified context.
