% Appendix Algebraic Model Counting

\chapter{Appendix Algebraic Model Counting} % Chapter title

\label{ap:amc} % For referencing the chapter elsewhere, use \autoref{ch:examples}
%----------------------------------------------------------------------------------------

The \ac{SAT} problem can be coined as the cornerstone of
almost all research relating intractable problems in Computer
Science. Its pure logical nature arises many problems related
to Combinatorial or decision problems \citep{Cook_1971,
levin1973universal}. While \ac{SAT} itself is a decision
problem, its optimization counterpart, the \ac{MAX-SAT}
problem, is also of great importance in many fields, from
\ac{AI} to Operations Research \citep{kolokolov2013analysis, gomes2006power}
and many other fields. On the other hand, the \ac{Sharp-SAT} problem elevates
the combinatorial nature of \ac{SAT} to a counting problem,
which increases its complexity and allows it as a reduction to
model many other counting problems \citep{valiant1979189}.

Another relevant generalization of \ac{SAT} is the \ac{WMC}, a
probabilistic generalization of \ac{Sharp-SAT}, where each
assignment of the variables in the propositional logic theory
has an associated weight, and the problem is to compute the
sum of weights of all satisfying assignments.

The great importance of \acl{WMC} in general is due to the fact
that it provides a framework for performing probabilistic
inference. This result stems from the possibility of reducing
probabilistic inference calls to \ac{WMC} on a propositional
knowledge base \cite{chavira2008probabilistic}. Specifically,
this approach derives from the possibility of encoding the
target probabilistic model, usually represented as a \ac{BN},
into a set of propositional clauses, where each clause is a
knowledge base in \ac{CNF}, and then assign weights to the
\ac{CNF} variables based on the network probabilities
\citep{darwiche2002logical,chavira2006compiling,
chavira2008probabilistic,sang2005performing, costa2012clp}.

This possibility of reducing general probabilistic inference
queries on \aclp{BN} to \ac{WMC} not only provides a declarative
method for encoding local structure (specific properties of
network parameters) and a powerful way for exploiting available
evidence \citep{chavira2008probabilistic}, but also enables the
use of highly optimized \ac{SAT} solvers or \ac{KC} techniques
\citep{kimmig2017algebraic, chavira2008probabilistic}, which
renders some of today's most efficient techniques for
probabilistic inference \citep{kimmig2017algebraic}.

Given the importance of all problems described above, a unified
framework capable of modeling all of them was proposed, called
\ac{AMC} \citep{kimmig2017algebraic}. Moreover, a second level
of this framework, naturally called \ac{2AMC}, was proposed
recently \citep{kiesel2022efficient}, which is capable of
reducing two main problems that this work aims to solve more
transparently: inference in \acl{PASP} under \textit{Credal}
and \ac{MaxEnt} semantics.

For the reader not familiar with \ac{SAT}-like problems, or even
\ac{WMC}, we recommend the reading of the appendix section
\ref{ap:sat}.

%------------------------------------------------

\section{Algebraic Model Counting}

Before defining the \ac{AMC} problem, we need to define the
type algebraic structure that is used to generalize tasks
such as \ac{SAT}, \ac{Sharp-P} (\ac{MC}) and \ac{WMC}. More
specifically, by generalizing these \ac{SAT}-like problems
to a more abstract \acl{WMC} over a different algebraic
structure, a \textit{comutative semiring}, we can define the
\ac{AMC} problem as follows \citep{kimmig2017algebraic}:

\section{Semirings}

\begin{definition}[Semiring]
    A semiring is an algebraic structure $(\mathcal{A}, \oplus,
    \otimes, e_{\oplus}, e_{\otimes})$, where addition $\oplus$
    and multiplication $\otimes$ are associative binary
    operations over the set $\mathcal{A}$, $\oplus$ is
    commutative, $\otimes$ distributes over $\oplus$,
    $e_{\oplus} \in \mathcal{A}$ is the neutral element of (the
    sum operator) $\oplus$, $e_{\otimes} \in \mathcal{A}$ is the
    neutral element of (the product operator) $\otimes$, and for
    all $a \in \mathcal{A}$, $e_{\oplus} \otimes a = a \otimes
    e_{\oplus} = e_{\oplus}$. In a commutative semiring,
    $\otimes$ is commutative as well.
\end{definition}

One particular notable semiring is the \textit{Tropical
Semiring}, in the context of \textit{idempotent analysis}, where
the sum operator $\oplus$ is defined as the minimum operator,
and the product $\otimes$ operator is defined as the usual real
sum. The set $\mathcal{A}$ in this case is the set of extended
real numbers, $\Set{\mathbb{R} \cup \Set{+\infty}}$. This
important semiring has various applications, with a special
focus in \textit{Tropical Analysis} and \textit{Tropical
Geometry}, and is named after Imre Simon's extensive work on
this specific semiring \citep{simon1988recognizable}. Within the
context of \ac{AMC}, the \textit{Tropical Semiring} can be used
to perform the \ac{MAX-SAT} problem, where the sum operator is
responsible for counting the number of satisfied clauses, and
the maximum operator is responsible for finding the maximum
number of satisfied clauses.

\section{Algebraic Model Counting}

\begin{definition}[\acl{AMC}]
    Given
    \begin{itemize}
        \item A propositional logic theory $T$ over a set of
        variables $\mathcal{V}$;
        \item A commutative semiring $(\mathcal{A}, \oplus,
        \otimes, e_{\oplus}, e_{\otimes})$; and
    \item A labeling function $\alpha: \mathcal{L}
        \rightarrow \mathcal{A}$, mapping literals $\mathcal{L}$
        of the variables in $\mathcal{V}$ to elements of the
        semiring set $\mathcal{A}$.
    \end{itemize}
    The \acl{AMC} problem is now defined as the computation of
    the following expression:
    \begin{displaymath}
        A(T) = \bigoplus_{I \in \mathcal{M}(T)} \bigotimes_{i
        \in I} \alpha(i),
    \end{displaymath}
    where $\mathcal{M}(T)$ denotes the set of models of $T$.
\end{definition}

To anyone familiar with introductory algebra, it is easy to see
that the definition above for the \ac{AMC} problem indeed
reduces many of the presented \ac{SAT} like problems. For
example, by setting the semiring to
$$ (\mathcal{A}, \oplus, \otimes, e_{\oplus}, e_{\otimes}) =
(\{\mathit{true}, \mathit{false}\}, \lor, \land, \mathit{false},
\mathit{true}),$$

and $\alpha$ maps positive literals to \textit{true} and negative
literals to \textit{false}, we can see that the \ac{AMC} problem
is capable of solving the \ac{SAT} problem. Moreover, by setting
a similar semiring

$$ (\mathcal{A}, \oplus, \otimes, e_{\oplus}, e_{\otimes}) =
(\mathbb{N}, +, \times, 0, 1),$$

and $\alpha$ to map literals to $1$ and $0$, if they
are positive or negative, respectively, we can see that the
\ac{AMC} problem \ac{Sharp-SAT}, also known as \acl{MC}.

Similarly, by only changing the set $\mathcal{A}$ of the
semiring to $\mathbb{R}^+$ (non-negative real numbers) and
$\alpha$ to also map literals to $\mathbb{R}^+$, we can see that
the \ac{AMC} problem is capable of solving the \ac{WMC} problem.

Moreover, by the results cited previously about \acl{BN}, one can see
that this also means that \ac{AMC} is also capable of reducing
\acl{PI} \citep{darwiche2002logical,chavira2006compiling,
chavira2008probabilistic,sang2005performing}. However, the
algebraic structure of \ac{AMC} makes it easy to show that it is
capable of modeling \ac{PI}: by using the same semiring used to
model the \ac{WMC} problem and just by changing the $\alpha$
function to represent a probability distribution over the
literals, setting $a(v) \in [0,1]$ and $a(\overline{v})$ to be
its complement, the \ac{AMC} problem is capable of modeling
general \acl{PI} queries.

In fact, \ac{AMC} is capable of modeling many other interesting
problems, such as $EXPEC$ (expectation), which allows one to
infer parameters in a \ac{FST} model w.r.t. to a given dataset.
Hence, the elements of the respective expectation semiring are
tuples of the form $(p, v)$, where $p \in [0,1]$ is a
probability of an arc of the \ac{FST} and $v \in \mathbb{R}$ is
the value of this respective arc. Then, the operations $\oplus,
\otimes$, neutral elements $e_{\oplus}, e_{\otimes}$, and the
labeling function $\alpha$ are defined as follows:

\begin{align*}
    (p_1, v_1) \oplus (p_2, v_2) & = (p_1 + p_2, v_1 + v_2), \\
    (p_1, v_1) \otimes (p_2, v_2) & = (p_1 \cdot p_2, p_1 \cdot
    v_2 + p_2 \cdot v_1), \\
    e_{\oplus} & = (0, 0), \\
    e_{\otimes} & = (1, 0), \\
    \alpha(a) & =
    \begin{cases}
        (p, 1) & \text{if } i = k, \\
        (p, 0) & \text{else.}
    \end{cases},
\end{align*}

where $a$ is the respective arc of the \ac{FST}, in relation to
the tuple $(p, v)$.

%Furthermore, the \ac{AMC} problem also is capable of reducing
%\ac{MAX-SAT} to it, by setting the semiring $(\mathbb{R}^+,
%\max, +, 0, 0)$, and the labeling function $\alpha(l)$, which
%maps literals to their respective weight contribution for all
%clause $c$ that the literal appears in. More specifically, the
%labeling function $\alpha$ is defined as follows:
%
%$$\alpha(l) = \sum_{c \in C : l \in c} \frac{w(c)}{|c|},$$
%
%where $C$ is the respective set of clauses of the propositional
%logic theory $T$, $w(c)$ is the weight of the clause $c$, and
%$|c|$ is the number of literals in the clause $c$.

Adapted from \citep{kimmig2017algebraic}, Table \ref{tab:amc}
summarizes the \ac{AMC} problem for different logical and
probabilistic inference tasks, describind in depth the
respective sets, operations and labeling functions associated
with each semi-ring.

\begin{table}
    \begin{center}
        \begin{tabular}{|c||c|c|c|c|c|c|c|}
            \hline
            $\mathrm{Task}$ & $\mathcal{A}$ & $\oplus$ & $\otimes$ &
            $e^{\oplus}$ & $e^{\otimes}$ & $\alpha(v)$ &
            $\alpha(\neg v)$ \\ \hline \hline
            $\mathrm{SAT}$ & $\set{\mathit{true},\mathit{false}}$ &
            $\vee$ & $\wedge$ & $\mathit{false}$ & $\mathit{true}$ &
            $\mathit{true}$ & $\mathit{true}$ \\ \hline
            $\mathrm{\#SAT}$ & $\mathbb{N}$ & $+$ & $\times$ & $0$ &
            $1$ & $1$ & $1$ \\ \hline
            $\mathrm{WMC}$ & $\mathbb{R}_{\geq 0}$ & $+$ & $\times$
            & $0$ & $1$ & $\in \mathbb{R}^+$ & $\in \mathbb{R}^+$
            \\ \hline
            $\mathrm{PI}$ & $\mathbb{R}_{\geq 0}$ & $+$ & $\times$ &
            $0$ & $1$ & $\in [0,1]$ & $1-\alpha(v)$ \\ \hline
        \end{tabular}
    \end{center}
    \caption{Examples of logical and probabilistic inference
    tasks that can be modeled as instances of the \ac{AMC}, and
    their corresponding semirings and labeling functions.}
    \label{tab:amc}
\end{table}

\begin{center}

\end{center}

Due to the importance of the semiring associated with \ac{SAT}
and \ac{PI}, we give a special name to them: the
\textit{Boolean} and the \textit{Probability} semi-rings.

Other useful problems that can be modeled by the \ac{AMC} task
are: \textit{sensitivity analysis} (SENS), \textit{probability
of most likely states} (MPE), \textit{shortest} and
\textit{widest path} (S-PATH and W-PATH, respectively),
\textit{fuzzy} and \textit{k-weighted} constraints (FUZZY and
k-WEIGHT, respetively), and $OBDD_<$ \textit{construction}
\citep{kimmig2017algebraic}.

\section{Second Level of Algebraic Model Counting}

As mathematical are prone to do, there is a generalization of
\ac{AMC}, called \acl{2AMC}, which is capable of modeling
problems where there is the need of a third operation, besides
$\oplus$ and $\otimes$. This third operation appears in many
applications, such as the \ac{MAP} probabilistical query, but is
also essential when modeling \ac{PASP} problems, as we will see
in more detail in the next chapter \ref{ch:pasp}.

Similarly to the \ac{AMC}, follows the definition of the
\ac{2AMC} problem \citep{kiesel2022efficient}:

\begin{definition}[\acl{2AMC}]
    Given
    \begin{itemize}
        \item A propositional logic theory $T$ over a set of
        variables $\mathcal{V}$;
        \item A partition of the variables in $T$,
        $(\mathbb{X}_I, \mathbb{X}_O)$;
        \item Two commutative semiring $S_j = (\mathcal{A}_j,
        \oplus_j, \otimes_j, e_{\oplus_j}, e_{\otimes_j})$, for
        $j \in \{I, O\}$;
        \item Two labeling function $\alpha_j: X_j \rightarrow
        \mathcal{A}_j$, for $j \in \{I, O\}$, mapping literals
        of the variables in $X_j$ to elements of the semiring
        set $\mathcal{A}_j$; and
        \item A weight transformation function $t:
        \mathcal{A}_I \rightarrow \mathcal{A}_O$ that respects
        $t(e_{\oplus_I}) = e_{\oplus_O}$.
    \end{itemize}
    Then the \acl{2AMC} problem is defined as the computation of
    the following expression:
    \begin{displaymath}
        2AMC(T) = \bigoplus_{\mathbf{a} \in A(X_O)}^O
                  \bigotimes_{a \in \mathbf{a}}^O \alpha_O(a)
                  \otimes_O t
                  \left (
                  \bigoplus_{I \in \mathcal{M}(T|\mathbf{a})}^I
                  \bigotimes_{i \in I}^I \alpha_I(i) \right ),
    \end{displaymath}
    where $A(X)$ denotes the set of assignments of $x$ to $X \in
    \mathcal{V}$, and $\mathcal{M}(T|a)$ denotes the set of
    models of $T$ given an assignment $a$.
\end{definition}

It is easy to see that \ac{AMC} is an instance \ac{2AMC}, where
the first partition is empty, $X_O = \emptyset$, and the weight
transformation is the identity function. Thus, the leftmost half
of the equation (all elements outside the parentheses) is equal
applying the identity function, and the rightmost half is
equivalent to the \ac{AMC}, because the assignment $\mathbf{a}$
inside the summation is empty, therefore, $M(T|\mathbf{a}) =
M(T)$. This implies that \ac{2AMC} tries to solve an instance of
\ac{AMC} over the variables present in the partition $X_I$ for
each possible assignment of the variables in $X_O$, applying a
weight transformation to the result. From the algebraic point of
view, this weight transformation function maps the result of the
\ac{AMC} inside the parentheses to the semiring $S_j$, used on
the left side of the equation. Thus, this transformation enables
one to solve a second \ac{AMC} instance over the variables in
$X_O$.

Therefore, analyzing the \ac{2AMC} problem, we can see that it
partitions variables in $\mathcal{V}$ into two sets, $X_I$ and
$X_O$, and  then solves an inner \ac{AMC} instance over the
variables presents in $X_I$ for each assignment to $X_O$. Then,
uses the results of this inner \ac{AMC} instance to solve a
second (level) \ac{AMC} instance over the variables in $X_O$.

Perhaps, a non-trivial example of a problem that can be modeled
as an instance of \ac{2AMC} is the \ac{MAP} query, which is
defined to be the most probable assignment $q$ given an
evidence $e$ (the assignment of the remaining variables that is
more likely given the evidence). Formally, given a propositional
logic theory $T$ over variables $\mathcal{V}$, a joint
probability distribution $P$ over $T$, a conjunction $e$ of
observed literals for the set of evidence atoms $E$, and a set
of ground query atoms $Q$: find the most probable assignment $q$ to $Q$
given the evidence $e$, with $R = \mathcal{V} \setminus (Q \cup
E)$

$$MAP(Q|e) = \argmax_q P(Q = q|e) = \argmax_q \sum_{\mathbf{r}
\in \mathcal{A}(R)} P(Q = q, e, R = r),$$

where $\mathcal{A}(R)$ denotes the set of assignments of $r$
to $R$.

Under the assumption that the distribution the Random Variables
respective to the variables in $Q$ are independent (the
distribution is factorized w.r.t. $Q$), we can see that \ac{MAP}
query consists of two steps:

\begin{enumerate}
    \item A first \ac{AMC} task of solving the sum over the
    truth values of the atoms in $R$, when considering fixed
    assignments to the atoms both in $E$ and $Q$; and
    \item A second \ac{AMC} task of determining the assignments
    to the variables in $Q$ that maximize the inner sum of the
    first \ac{AMC} task.
\end{enumerate}

This intuitive description of the \ac{MAP} query illustrates how
it could be modeled as an instance of the \ac{2AMC} problem. By
having the partitions of variables to be $X_O = Q$ and $X_I =
\mathcal{V} \setminus Q = R \cup E$, we have a compatible
partition of the variables in $\mathcal{V}$ w.r.t. the \ac{MAP}
problem.

With the partition of variables defined, there is still the need
to define the semirings, labeling functions, and the weight
transformation function $t$. Since the rightmost half of the
equation is w.r.t. to summing the probabilities over assignments
$r$ to $R$, the semiring used for this part of the equation is
$S_I = ([0, 1], +, \times, 0, 1)$ (the same semiring used for
\ac{PI}); and the labeling function $\alpha_I$ maps literals
present in $e$ to $1$ and $0$ to their negation, $P(r)$ and
$1 - P(r)$ to the positive and negative literals present in $R$
(thus, the probability of assignments that are not "compatible"
with $e$ is zero).

On the other hand, the leftmost half of the equation represents
a maximization of the truth values of the atoms in $Q$, given
the result of the inner \ac{AMC} instance. Therefore, the
semiring used for this part of the equation is $S_O = (R^+
\times 2^{|Q|}, \oplus_{\mathrm{argmax}},
\otimes_{\mathrm{argmax}}, (0, \emptyset), (1, \emptyset))$,
where

$$(p_1, q_1) \oplus_{\mathrm{argmax}} (p_2, q_2) =
\begin{cases}
    (p_1, q_1) & p_1 > p_2 \\
    (p_2, q_2) & p_1 < p_2 \\
    (p_1, q_{\mathrm{min}}) & p_1 = p_2
\end{cases}
$$

and $q_{\mathrm{min}}$ represents the smallest assigment of
literals, w.r.t. a predefinied lexycographic order of the
variables $Q$; and $\otimes_{\mathrm{argmax}}$ is defined as the
product between the probabilities of the assignments, and the
union of the assignments of the literals, $(p_1, q_1)
\otimes_{\mathrm{argmax}} (p_2, q_2) = (p_1 \cdot p_2, q_1 \cup
q_2)$. Moreover, to complete the definition of the this
leftmost \ac{AMC} instance, the labeling function $\alpha_O(l)$
is defined to be: $(p, \{l\})$ for the positive literals; and
$(1 - p, \{l\})$ for the negative ones.

Finally, to completely describe \ac{MAP} as an instance of the
\ac{2AMC} problem, we construct the function $t(p)$, where $p$
is the resulting probability of the inner (rightmost) \ac{AMC}
task, as being $t(p) = (p, \emptyset)$. This weight
transformation function takes the sum of the probabilities as
input and just returns the Cartesian Product of the
probability and the empty set, which is the neutral element of
the $\oplus_{\mathrm{argmax}}$ operator. Next, the $\otimes_{
\mathrm{argmax}}$ will take the product of the probabilities an
assignment of a variable in $Q$ and the result probability of
the inner \ac{AMC}. In other words, the probability $P(q)$,
where $q \in \mathbf{q}$ and $\mathbf{q}$ is the assignment of
the variables in $Q$, is multiplied by the sum of probabilities
conditioned on the assignment $q$ and evidence $e$. The second
part of the tuple is the union between $q$ and $\emptyset$,
which is equal to $q$. Thus, the $\otimes_{\mathrm{argmax}}$
applies the Bayes Rule to the result of the inner \ac{AMC},
multiplying the probability of the assignment of the variables.
Therefore, the last step, $\oplus_{\mathrm{argmax}}$, takes the
assignment that maximizes the probability of the computation
performed by the $\otimes_{\mathrm{argmax}}$ operator, which
results in the most probable assignment of the variables in $Q$.
