% Chapter 1

\chapter{Introduction}

\label{ch:introduction}

Argumentation is a fundamental aspect of human communication, arising in
contexts ranging from everyday discussions to formal debates and legal
reasoning. The importance of argumentation extends beyond mere persuasion; it is
central to critical thinking, decision-making, and the construction of
knowledge. Hence, understading how arguments are structured, how they interact
and how they can be evaluated is a key challenge for both social sciences and
Artificial Intelligence (AI)

An area of growing interest within AI is \emph{argumentation mining}, which
seeks to automatically identify and extract argumentative structures from
natural language texts \citep{stab2017parsing}. Therefore, this field can be
greatly enhanced by Neuro-Symbolic AI techniques, that combine the strengths of
symbolic reasoning and connectionist learning. More specifically, connectionist
models excel at processing unstructured data, such as text, extracting
argumentative components and their relations. On the other hand, symbolic
reasoning provides a framework for robustly constraining and interpreting these
components, either in inference or learning.

Core tasks in argumentation mining may include: separating argumentative from
non-argumentative spans, classifying argument components, and predicting the
support or attack relations between them. These tasks share tight interectios,
where errors in earlier stages, such as claim identification, can easily
propagate and degrade the coherence of the recovered  argumentative graph
\citep{stab2017parsing}. By optimizing these tasks jointly within a structured
framework, with global constraints, we can mitigate this error propagation and
improve the system's overall performance.

One of the main approaches for this task was proposed in \citep{stab2017parsing},
where the authors introduce a joint model for argumentation mining based on
Integer Linear Programming (ILP). Their model combines local classifiers for
argument component identification and relation prediction with global constraints
that enforce the well-formedness of the argumentative structure. This joint
model demonstrates improved performance over pipeline approaches, highlighting
the benefits of integrating local predictions with global reasoning. On the
other hand, the ILP-based approach struggles to naturally express uncertainty
in the predictions, and its reliance on discrete optimization can limit its
scalability and flexibility.

More recent approaches focused on leveraging Probabilistic Logic Programming
(PLP) systems, such as ProbLog \citep{fierens2015}, to model argumentation
mining. These approaches benefit from the declarative nature of PLP, allowing
for more interpretable models that can naturally incorporate uncertainty and
logical reasoning, without the need of expert knowledge to define complex ILP
constraints. For instance, the framework proposed by \citep{cerveira2020joint}
leverages DeepProbLog \citep{manhaeve2018deepproblog} to combine neural
networks to handle the connectionist part of the task, with ProbLog to model the
symbolic reasoning about argumentative structures. This hybrid approach allows
for end-to-end learning while maintaining the interpretability and
expressiveness of the symbolic component.

Another relevant work is \citep{totis2023smproblog}, which extends the
ProbLog framework by incorporating the Stable Model semantics from Answer Set
Programming (ASP). This extension, known as \textsc{smProbLog}, enables the
modeling of more complex argumentative patterns, such as negative cycles, which
may arise in contradictory or even vague argumentative contexts. By adopting
stable model semantics, \textsc{smProbLog} provides a more expressive framework
for argumentation, allowing for richer representations of argumentative
structures and their interactions.

This dissertation examines how probabilistic logic programming, through the usage of
knowledge compilation compilation techniques, can build end-to-end
Neuro-Symbolic pipelines to learn rich argumenatative structures from data,
detecting how claims, premises, and attacks interplay in text. We aim to explore
how Deep Probabilistic Answer Set Programming can bridge the gap
between symbolic argumentation theory and data-driven argumentation mining. By
embedding argumentative structures directly in probabilistic circuits, we aim to
support transparent reasoning, quantified uncertainty, and efficient integration
with neural learning.

This works aims to explore the following topics:
\begin{itemize}
  \item How to formalize argumentation mining tasks as PASP programs that
  capture the mutual influence of claims, premises, and attacks while retaining
  probabilistic semantics for uncertain or conflicting evidence.
  \item The development of knowledge compilation pipeline that encodes PASP
  programs into structured probabilistic circuits amenable to tractable inference.
  \item Techniques for converting such circuits into differentiable structure
  that may be used in neuro-symbolic learning strategies that exploit the compiled
  circuits to align neural predictions with argumentative constraints, paving the
  way for scalable argumentation mining under uncertainty.
\end{itemize}

The remainder of this document is organized as follows. The first three chapters
Chapters \autoref{ch:sat}, \autoref{ch:amc}, and \autoref{ch:asp} reviews basic
concepts in propositional satisfiability, model counting over semirings, and
logic programming foundations. After, in Chapter \autoref{ch:contribution},
we introduce how PASP can used to encode argumentative structures. Building on
this, we introduce basic concepts related to Knowledge Compilation and the
proposed pipeline for end-to-end Neuro-Symbolic learning for argumentation
mining. Finally, we situate our approach within existing work, and discuss
future directions.
